Suppose $\cat{C}$ is a category and $(\cat{E},\otimes,I)$ is a strict
monoidal category.

\begin{definition}[Graded $\func{F}$-Algebra]
    For a functor $\func{F} : \cat{E} \times \cat{C} \mto \cat{C}$, a
    graded $\func{F}$-algebra is a pair $(\func{A},h)$ that consists
    of a functor $\func{A} : \cat{E} \mto \cat{C}$ and a family $h$ of
    morphisms:
    \[ 
        h_{m,n} : \func{F}(m,\func{A}(n)) \mto \func{A}(m \otimes n)
    \]
    A \emph{homomorphism} between two graded $\func{F}$-algebras
    $(\func{A},h)$ and $(\func{B},h')$ consists of a morphism
    \[
        \alpha : (\func{A},h) \mto (\func{B},h')
    \]
    is defined as a natural transformation $\alpha : \func{A}_1 \mto
    \func{A}_2$ such that:
    \[
        \func{F(m,\alpha_n)} ; h'_{m,n} = h_{m,n} ; \alpha_{m \otimes n}
    \]
\end{definition}

\begin{definition}
    If the category of graded $\func{F}$-algebras has an initial object, then
    we call this a \emph{graded initial} $\func{F}$-algebra denoted
    by $(\mu\func{F}, \mathsf{in})$.  That is, for any other
    $\func{F}$-algebra $(\func{A},h)$ there must be a unique morphism 
    $\alpha : (\mu\func{F}, \mathsf{in}) \mto (\func{A},h)$, but this
    implies that for any object $n$, $\alpha_n : \mu\func{F}(n) \mto
    \func{A}(n)$ is unique and $\mu\func{F}(n)$ is an initial object
    in $\cat{C}$.
\end{definition}

\begin{lemma}[From Structures to Homomorphisms]
\label{lemma:from_structures_to_homomorphisms}
Given any graded $\func{F}$-algebra $(\func{A},h)$, the structure map
$h$ is also a homomorphism between $\func{F}$-algebras
$(\func{F}(m,A(-)),\func{F}(-,h_{m,-}))$ and $(\func{A}(m \otimes -)),h_{-, m \otimes -})$.
\end{lemma}
\begin{proof}
This proof holds trivially by writing out the commutative square for
the $\func{F}$-algebra homomorphism.
\end{proof}

\begin{lemma}[Graded Lambek's Lemma]
\label{lemma:graded_lambeks_lemma}
If $(\func{A},h)$ is a graded initial algebra of $\func{F}$, 
then for any object $m$, 
$\func{A}(m \otimes -) : \cat{E} \mto \cat{C}$ is isomorphic to 
$\func{F}(m, A(-)) : \cat{E} \mto \cat{C}$ via $h_m$.
\end{lemma}
\begin{proof}
Suppose $h_{m,n} : \func{F}(m,A(n)) \mto \func{A}(m \otimes n)$ is an initial
algebra structure for any $m$ and $n$.  
Now define an algebra structure:
\[
\func{F}(m',h_{m,n}) : \func{F}(m',\func{F}(m,A(n))) \mto \func{F}(m',\func{A}(m \otimes n))
\]
Then by initiality there exists an $\func{F}$-algebra homomorphism 
$$i_{m} : \func{A}(m \otimes -) \mto \func{F}(m,\func{A}(-))$$ such that:
$$\func{F(m',i_{m,n})} ; \func{F}(m',h_{m,n}) = h_{m',(m \otimes n)} ;
i_{m',(m \otimes n)}$$
We also know that $h_{m} : \func{F}(m,A(-)) \mto \func{A}(m \otimes -)$ is
itself a graded $\func{F}$-algebra homomorphism
(Lemma~\ref{lemma:from_structures_to_homomorphisms}). Thus, since we
know that $\func{A}(m \otimes n)$ is an initial object by definition and
assumption that $\func{A}$ is a graded initial object, and hence, 
$i_{m,n}; h_{m,n} = \id_{m \otimes n}$.  

Next we know that $i$ is a graded $\func{F}$-algebra homomorphism which
implies $$\func{F}(m,i_{n,I});\func{F}(m,h_{n,I}) =
h_{m,n};i_{m,n}$$ but again by initiality we know that
$$\func{F}(m,i_{n,I});\func{F}(m,h_{n,I}) =
\id_{\func{F}(m,\func{A}(n))}$$
Therefore, $i$ is the inverse of $h$ and we obtain our result.
\end{proof}

\begin{definition}[Graded Folds]
\label{def:graded_folds}
Suppose $(\mu\func{F},\mathsf{in})$ is a graded initial
$\func{F}$-algebra. Then the unquie map between $\mu\func{F}$ and any
other graded $\func{F}$-algebra $(\func{A},h)$ is the \emph{fold} for
$\mu\func{F}$ and is denoted by 
\[
    \mathsf{fold}(h) : \mu\func{F} \mto \func{A}
\]  
Furthermore, we know that the following must hold:
\[
\mathsf{in}_{m,n};\mathsf{fold}(h)_{m * n} = \func{F}(m,\mathsf{fold}(h)_n);h
\]
\end{definition}

\begin{definition}[Graded Forgetful Limits]
\label{def:forgetful_limits}
There is a forgetful functor from the category of graded
$\func{F}$-algebras and the functor category $[\cat{E},\cat{C}]$ and
their natural transformations. This functor is defined as follows:
\[
    \begin{array}{lll}
        \func{U}_\func{F}(\func{A},h) = \func{A}\\
        \func{U}_\func{F}(\alpha) = \alpha
    \end{array}
\]
Given an object of $[\cat{E},\cat{C}]$, say $\func{X}$, then a
\emph{$\func{U}_\func{F}$-cone for $X$} comprises, for every graded
$\func{F}$-albebra $(\func{A},h)$, a natural transformation 
$v_{(\func{A},h)} : \func{X} \mto \func{A}$ in $[\cat{E},\cat{C}]$
such that, for every graded $\func{F}$-algebra homomorphism $\alpha :
\func{A} \mto \func{B}$, we have $v_{(\func{B},g)} = \alpha;v_{(\func{A},h)}
$.  We denote these cones by $(\func{X},v)$ and call $\func{X}$ its
\emph{vertex} and $v_{(A,h)}$ the \emph{projection} from $\func{X}$ to
$\func{A}$. 

A \emph{$\func{U}_\func{F}$-cone morphism} $g : (\func{X},v_1) \mto
(\func{Y},v_2)$ is a natural transformation $g : \func{X} \mto
\func{Y}$ such that for any graded $\func{F}$-algebra $(A,h)$, we have
$g;v_2 = v_1$. A \emph{$\func{U}_\func{F}$-limit} is a
$\func{U}_\func{F}$-cone to which there is a unique
$\func{U}_\func{F}$-cone morphism, call the \emph{mediating morphism},
from any other $\func{U}_\func{F}$-cone.
\end{definition}

\newcommand{\UFunc}[1]{\func{U}_{\func{#1}}}
\newcommand{\UCone}[1]{$\UFunc{#1}$-cone}
\newcommand{\ULimit}[1]{$\UFunc{#1}$-limit}

\begin{lemma}[Forgetful Limits are Unique]
\label{lemma:forgetful_limits_are_unique}
If $(\func{X},v)$ is a $\func{U}_{\func{F}}$-limit, then it is unique
up to isomorphism.
\end{lemma}
\begin{proof}
Suppose $(\func{Y},v')$ is another \ULimit{F}. Thus,
there is a unique \UCone{F} morphism $i : \func{Y} \mto \func{X}$ such
that $i;v = v'$. But, there must also be a unique \UCone{F} morphism $j :
\func{X} \mto \func{Y}$ such that $j;v' = v$.  But, by substitution
$i;j;v' = v'$ and $j;i;v = v$, but these in addition to the assumption
that both $i$ and $j$ are unique imply that $i;j = \id_\func{Y}$ and
$j;i = \id_\func{X}$, and thus $i$ and $j$ are inverses of each other. 
\end{proof}

\subsection{Interpretation}
\label{subsec:interpretation}
\begin{itemize}
\item A data type is seen as the carrier of the initial algebra of a
higher-order functor with type $(|\cat{C}| \to \cat{C}) \to |\cat{C}|
\to \cat{C}$.
\item Constructors have return types of the form $\func{G} (h\,a)$,
but initial algebra semantics requires them to be of the form
$\func{G}(a)$. To get an equivalent data type with constructors whose
return types have this form we use left Kan extensions.
\end{itemize}
% subsection interpretation (end)